% Chapter Template

\chapter{Future Work} % Main chapter title

\label{Chapter9} % Change X to a consecutive number; for referencing this chapter elsewhere, use \ref{ChapterX}

This project was focused on demonstrating the advantages achieved by migrating a big data analytics system to a container-orchestration system. My system prototype has an impressive set of features but it is, however, just a proof of concept. As such, there are many areas for improvement including resource usage, security, database optimization, and extensibility.

With respect to resource usage, there is room for improvement on understanding the usage constraints of each of my system components. If I could identify upper and lower bounds for each component, I could provide the Kubernetes scheduler with even more information in which to make optimization decisions and container allocations.

Another feature that is needed is a centralized system for authentication and authorization. My research prototype deliberatly avoided imposing any security measures as our focus was on what was feasible with respect to reliablity and scalability. However, if this system were to be transitioned into a production environment, then a wide variety of security-related techniques would need to be adopted: encryption of data, securing of individual components, the addition of user and system roles and authorization of those roles to access particular types of data and engage in particular types of operations.

There is plenty of room for improvement with respect to the structure of my Cassandra tables. For my thesis work, I used a  naive approach to tweet storage. It works well for what I needed to do but there are plenty of ways in which it could be improved. A way to make this better is by upgrading event\_name to be a partition key. This would allow Cassandra to cluster tweets that belong to the same event improving all queries that are event-based. In addition, I would need to add logic in the partition assignment for the tweet normalizer, as Cassandra limits the amount of rows a partition can contain and we need a way to make it easier for data to be distributed across all nodes. We could do this by assigning a random partition number when the tweet normalizer starts and choose a new value every one hundred thousand stored tweets. This would need to be studied more carefully in the future.

With respect to system extensibility, there are a few ways that the prototype could be improved. For example, I could add a real-time query resolver for a very specific query by plugging it into the raw\_tweets queue and make it analyze the data. Or I could add other systems for specific queries like Elastic search for full-text search or analysis. In addition we could extend the system to include better collaboration tools like a notification system plugged into the event\_updates queue.

As one can see, there are many possibilities for improvement and due to the benefits of container-orchestration system, many of these improvements can be done in isolation without the need to upgrade all of the components at once.
