% Chapter Template

\chapter{Introduction} % Main chapter title

\label{Chapter1} % Change X to a consecutive number; for referencing this chapter elsewhere, use \ref{ChapterX}



Big Data Analytics is quickly becoming one of the most emergent fields of work. We live in a highly connected society that generates tons of data. Every day, 2.5 Exabytes are moved around the Internet. In addition, it’s expected that by 2020, 44 thousand million gigabytes will be the amount of data stored on earth. This constant growth is probably one of the biggest challenges we face in the computer science world. As a consequence, new systems need to be designed to support all of this demand, keeping the response time as well. 
 
On the other hand, containerization is becoming a trend nowadays. Thanks to companies like Docker and its Docker Engine, containers are becoming a de facto standard for software development. It’s ease of use and consistent behaviour between systems makes it attractive to developers, as they prefer avoiding difficult and long system setup every time they need to run an application.
 
With this trend, complexities arrived: difficulties to configure the network and interaction between containers, difficulties to deploy and release new versions and way more. That’s when container orchestrated system arrived. They are an abstraction layer over the container management. This systems are in charge of coordinating containers, creating them, destroying them: all the container lifecycle is controlled by the system.
 
As this systems are becoming more and more popular, software architecture will need to adapt to embrace the new possibilities that this systems provide. Some concepts that previously were close to impossible to do, now are available to any software architect. Applications no longer need to have an static infrastructure, container orchestrated systems can grow and decrease on demand. Systems can be splitted in small microservices, there’s no need to merge everything a big monolith application anymore.
 
In addition, cloud providers are adding support for container orchestrated systems. This makes it even easier to deploy complex infrastructures. Also, you can design systems independently of the providers. 
 
It also helps to find new ways to solve the problem of parallelism and scaling, making it easier to deploy new machines opens the door to new possibilities to have infrastructure on demand or to scale automatically.
 
