% Chapter Template

\chapter{Introduction} % Main chapter title

\label{Chapter1} % Change X to a consecutive number; for referencing this chapter elsewhere, use \ref{ChapterX}

Big Data Analytics is a research area undergoing intense interest and rapid development. Data is being generated by sensors and computing systems at an ever increasing rate. Indeed, it is estimated that every month, seventy-two petabytes of information are moved around the Internet \parencite{ciscoreport}. This amount is expected to grow to 232 petabytes/month by 2021 \parencite{ciscoreport}. This rapid growth is a significant challenge to the designers of software systems and to the field of computer science in general. New software infrastructure will need to be designed to support this demand and to extract useful information from it. In addition, a wide range of techniques and technologies must be mastered to keep analysis time low at this scale.

One technique that can be useful in addressing these challenges is \textit{containerization}; indeed, tools and technologies that provide containerization services are seeing increased interest and development. The ability to run software in separate environments has been a revolution for the tech industry. Containerization provides a layer of abstraction for a set of services that allow software engineers to avoid long installation processes on each machine a software development team needs to run their software. Containerization is similar to \textit{virtualization} but optimized to use system resources instead of simulated resources. Put another way, a \textit{virtual machine image} contains everything needed to simulate a separate operating system. If you have multiple copies of that image, each image contains a complete copy of the operating system and all other components needed to run your software, wasting valuable disk space and consuming extra memory at run-time. With containerization, each container image contains just the software that you want to deploy and its required packages and the containerization run-time system provides a unified operating system for running all containers. Thanks to companies like Docker, containers are becoming a de facto standard for software development and deployment.\footnote{\href{https://docker.com}{https://docker.com}} The ease of use and consistent behaviour between machines provided by containerization makes it an attractive option for developers.

However, this approach brings with it a few complexities: network configuration, interaction between containers, system updates, and more. To solve these complexities, \textit{container-orchestration} systems were developed. These systems provide another abstraction layer on top of containers, taking over the management of an underlying containerization technology. In particular, the container-orchestration system is in charge of coordinating containers, creating them, destroying them, etc. The entire container life cycle is controlled by the orchestration system.

As these orchestration systems become more popular, software architects must create new software architectures that embrace the  possibilities that these systems provide. Some features that previously were difficult to achieve are now available to any software architect. Applications no longer need to have a static infrastructure: container-orchestrated systems can grow and decrease system resources on demand. Systems can be split into small microservices and deployed across a cluster of machines; there is no need to merge components into a big monolithic application.

Furthermore, cloud providers are adding support for container-orchestrated systems, making it straightforward to deploy complex infrastructures. Due to the similarities between different cloud offerings, one can design this new style of software system independent of a specific provider. One benefit to container-orchestrated systems is that it provides a means for finding new ways to solve problems associated with making use of parallelism and scaling a software system to handle ever increasing amounts of data. If properly designed, surges in demand can be met by spinning up additional copies of core system services balanced across the cluster of machines reserved from a cloud provider.

The goal of my thesis is to explore how this new style of software architecture can be used to deploy software infrastructures that are used to perform \textit{big data analytics}. As part of my thesis work, I migrate an existing big data infrastructure to make use of the techniques and technologies offered by container-orchestrated systems and compare the new version of the system with its previous incarnation along a number of dimensions.
