% Chapter Template

\chapter{Problem Statement} % Main chapter title

\label{Chapter3} % Change X to a consecutive number; for referencing this chapter elsewhere, use \ref{ChapterX}

The goal of my thesis is to explore the benefits and limitations in using container-orchestration systems to build and deploy software systems that engage in big data analytics. To perform this exploration, I will be redesigning the Project EPIC software infrastructure as a set of microservices---each inside a separate Docker container---that are deployed on a cluster of cloud-based machines via a container-orchestration system. The current infrastructure was manually deployed on a set of physical machines in a local data center and was not developed using microservices or containerization. My hypothesis is that I’ll be able to achieve greater scalability and reliability with the new architecture with significantly reduced maintenance costs. I will also explore whether I am able to achieve greater query flexibility and overall performance using this new style of software infrastructure. My specific research questions are:

\begin{enumerate}
	\item What advantages and/or limitations will the new Project EPIC infrastructure have with respect to its predecessor?	
	\begin{enumerate}
		\item Is it more reliable? If so, why?
		\item Is it more scalable? If so, why?
	\end{enumerate}
	\item Does the new infrastructure have lower maintenance costs than the existing infrastructure?
	\begin{enumerate}
		\item Is it easier to deploy?
		\item Is it easier to upgrade?
		\item Is it more resilient to failures? If so, how?
	\end{enumerate}
\end{enumerate}


