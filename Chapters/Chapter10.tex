% Chapter Template

\chapter{Conclusions} % Main chapter title

\label{Chapter10} % Change X to a consecutive number; for referencing this chapter elsewhere, use \ref{ChapterX}

Container-orchestrated technologies make it easier to develop big data analytics systems. Their abstraction layer allows for a separation of responsibilities between developers and system operators, allowing them to work together without too much overlap in their work. This provides potential for innovation within big data analytics, as improvements can be developed separately for infrastructure and software. It also opens a gate to new approaches to the state of these systems, extracting the state out of the individual system components and into the surrounding container-orchestrated system and the messages that get passed between the componets.

In my work, using container-orchestration systems to recreate as much of the core functionality of the existing Project EPIC infrastructure has allowed me to develop a system that has proven to be easier to scale. It also has produced a system with greater reliablity by moving that responsibilty out of the big data system itself and into the orchestration system. My prototype has also proved to be easier to mantain, as components are smaller and it is easier to adopt a more continous development and deployment cycle for each system component. In addition, container systems like Docker allow developers to focus more on an application's logic instead of worrying about deployment infrastructure.

In conclusion, I have demonstrated that container-orchestration systems are a great option for developing big data analytics infrastructures that require flexible scaling and high reliability.
